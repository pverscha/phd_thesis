\chapter*{Summary}

Proteins are ubiquitous.
They are an essential part of all life on earth and they are responsible for a whole range of vital functions.
Enzymes, for example, take care of converting nutrients into energy and other building blocks required for a properly functioning body.
Or hormones, another class of proteins, take care of the communication between different parts and organs in our body and regulate essential processes.
Not only humans, but all living organisms on our planet depend on proteins for their survival and efficient protein research is thus of significant importance.

Proteins are so-called macromolecules (i.e. very big molecules), constructed from smaller amino acids.
These amino acids are floating around in our body and are connected in a specific sequence in order to form proteins with a specific function and structure.
The exact sequence of these amino acids is what determines the function of the proteins and is recorded as part of our DNA.
A small change to an organism's DNA could lead to some proteins not being able to be produced anymore, or altering their functionality.
Since the first discovery of proteins in nature, a lot of research around these big molecules has been conducted and allowing researchers to compile a big protein database where they keep track of which protein is associated with which organism and what its (known) function is.

With the help of a mass spectrometer, a very expensive and complex measuring device, and a series of advanced computer analyses, it is possible to derive which protein sequences are present in a specific environment (e.g. blood, stool, soil, etc.).
Subsequently, the protein sequences can then be looked up in the protein reference database and a summary of all identified organisms in the ecosystem under study (as well as the functions that they are responsible for) can be compiled.

In order to make this process as easy as possible, we started the development of Unipept at Ghent University.
Unipept is a software application that processes a series of protein fragments (i.e. peptides) and looks up each of these fragments in the protein reference database in order to construct a taxonomic and functional profile of an ecosystem under study.
The taxonomic profile of a sample, together with a collection of interactive data visualizations, will help researchers to find out \textbf{who} is present in an ecosystem.
In order to increase insight into \textbf{what} these organisms are doing at a specific point in time, researchers can take a look at the functional profile of a sample (as generated by Unipept).
This reasoning clearly explains what the advantage is of studying proteins instead of looking at the DNA found in a sample.
In stead of only figuring out \textit{who} is present, we can also find out \textit{what} these organisms are doing.

Over the last 10 years, a lot of technical advances have been made in the field of mass spectrometers.
This allows protein researchers to more easily analyse large samples in one go and generate a larger amount of data that needs to be processed by Unipept subsequently.
Because Unipept was initially developed as a web application and thus always relies on a web browser (such as Google Chrome), it becomes harder and harder to keep up with the advances in mass spectrometry technology and the huge amount of data that comes with it.
In \autoref{chapter:unipept_desktop_v1} you can read how I tackled this problem by developing the novel Unipept Desktop application.
The first version of the Unipept Desktop application allows to process samples that are up to 10 times larger than before (containing up to half a million peptides) and to compare analysis results with each other.
Unipept Desktop v1.0 also allows, for the first time, to organize samples into projects and studies (allowing researchers to link similar experiments).

In \autoref{chapter:unipept_desktop_v2} I go one step further and present to you version 2.0 of the Unipept Desktop application.
This version is the first one to provide support for the analysis of \textbf{proteogenomics} samples.
Proteogenomics is a novel research discipline that is currently emerging and consists of combining the information from a genetic experiment (derived from DNA analyses) with the information that is extracted from a protein sample.
Proteins that are produced by distinct organisms can still be very similar, making it impossible for Unipept to distinguish between them.
Instead of returning a list of all exact organisms that are detected, Unipept will typically return a list of organism groups that potentially \textit{can} be present in the ecosystem under study.
Sometimes, this information is sufficient for researchers to continue with their experiments, but often it occurs that the functional profile of the ecosystem is not informative enough.

In the study of proteogenomics, researchers are first going to perform a genetic experiment in order to find out which organisms are present in an ecosystem and in order to decrease the size of the protein search space.
Only the proteins of those organisms that were identified in the first step will be considered during the subsequent protein matching step which can then potentially increase the resolution of the taxonomic and functional profile as generated by Unipept.

A lot of different functional classification systems for proteins exist and Unipept provides extensive support for three of them.
The Enzyme Commission numbers (EC-numbers), Gene Ontology terms (GO-terms) and InterPro entries all have their own advantages and disadvantages and typically each focus on a distinct class of proteins.

In \autoref{chapter:unipept_cli_v2}, I explain how we have expanded the \textbf{Unipept API and CLI} with support for functional annotations.
The API (Application Programming Interface) is a collection of resources that are provided by Unipept and that can be used by third-party application to integrate some of Unipept's analysis features.
Unipept's CLI (Command Line Interface) is a separate software application that does not provide a graphical user interface, but that can easily be plugged into existing analysis pipelines and allows processing larger samples.

One of Unipept's major strengths is the collection of interactive data visualizations that it provides which each increase the insight of reseachers into the taxonomic and functional composition of a sample.
Each of these visualizations have been developed in-house and are not only suitable for visualizing protein analysis results, but can be applied to a wide range of different data sets.
In \autoref{chapter:unipept_visualizations} I talk briefly about the development of these visualizations and their public availability as a software library.

Unipept will typically report a whole collection of GO-terms for each protein sample, which opens up the possibility for researchers to compare these samples with each other in order to find out how \textbf{similar} they are.
In \autoref{chapter:megago} I explain how I, together with a group of researchers from all over Europe, developed a similarity metric for GO-terms.
This metric produces a value between $0$ and $1$ everytime it is given two sets of GO-terms.
The closer this values is to $1$, the higher the similarity between these two sets of GO-terms.

Next to this listing of improvements that I have made to Unipept over the years, I briefly talk about a number of different projects that I was deeply involved with during my PhD in \autoref{chapter:other_projects}.
Finally, a discussion about what the future might hold for Unipept, can be found in \autoref{chapter:future_work}.
