\chapter*{List of conferences and research stays}

\begin{large}\textbf{\textsf{EuBIC Developer's Meeting 2020}}\end{large} \\
\faCalendar ~ \textsf{January, 2020} \hfill \faGlobe ~ \textsf{Nyborg, Denmark} \\
Together with Henning Schiebenhoefer, Tim Van Den Bossche and Bart Mesuere, I hosted a hackathon called ``Mapping proteins to functions: method and benchmark development``.
During this hackathon, we developed a tool called ``MegaGO`` which has also been published in the Journal of Proteome Research.
The result of this hackathon was a nice software package consisting of a web application, a command line tool and an application programming interface that is publicly accessible.

\begin{large}\textbf{\textsf{VIB: Research Software Developers Day}}\end{large} \\
\faCalendar ~ \textsf{December, 2020} \hfill \faGlobe ~ \textsf{Online} \\
The Research Software Developers Day, organised by VIB, was an online conference that consisted of software developers presenting their workflow and ideas about the development of research software.
I was one of the presenters on this day and performed a 20-minute talk about the development of the Unipept Desktop application. This talk was followed by 10 minutes of questions and discussion afterwards.
A recording of the talk is available on YouTube: \href{https://youtu.be/2ftKfJkcJeY}{https://youtu.be/2ftKfJkcJeY}.

\begin{large}\textbf{\textsf{Online Metaproteomics Symposium}}\end{large} \\
\faCalendar ~ \textsf{June, 2021} \hfill \faGlobe ~ \textsf{Online} \\
This small online symposium mainly consisted of a quick overview and update of recent developments in the field of metaproteomics and an introduction to the recently started Metaproteomics Iniative.
I was only a spectator at this online conference and did only passively participate in the activities.

\pagebreak

\begin{large}\textbf{\textsf{Fourth Metaproteomics Symposium}}\end{large} \\
\faCalendar ~ \textsf{September, 2021} \hfill \faGlobe ~ \textsf{Luxembourg, Luxembourg} \\
The aim of this symposium was to provide a platform for the participants to share their latest results in their respective fields using metaproteomic methods as well as discussing recent technologic innovations and presenting newly developed bioinformatic tools.
I presented the Unipept Desktop tool at this conference during a 20-minute presentation (with a 10 minute questions session afterwards).

\begin{large}\textbf{\textsf{EuBIC Winter School 2022}}\end{large} \\
\faCalendar ~ \textsf{March, 2022} \hfill \faGlobe ~ \textsf{Lisbon, Portugal} \\
The EuBIC-MS Winter School on computational mass spectrometry (MS) takes place every two years.
Its aim is to bring together the users and developers of computational mass spectrometry tools, as well as academia and industry.
The winter school started with an educational day dedicated to workshops and trainings in established computation MS tools and workflows.
The following days, internationally renowned invited speakers gave lectures and practical workshops covering the aspects of identification, quantification, result interpretation, and integration of MS data.
I presented a poster during this conference and participated in the workshops and talks given during this week.

\begin{large}\textbf{\textsf{Research stay at the Bundesanstalt für Materialforschung und -prüfung (BAM)}}\end{large} \\
\faCalendar ~ \textsf{September - October, 2022} \hfill \faGlobe ~ \textsf{Berlin, Germany} \\
Together with Tanja Holstein, I worked on a project to integrate PepGM with Unipept.
PepGM is a probabilistic graphical model for taxonomic inference of proteome samples with associated confidence scores.
It started out as a tool for the analysis of viral proteome samples and by better integrating Unipept with PepGM, we were able to expand the tool with support for proper metaproteomics datasets.

\begin{large}\textbf{\textsf{HUPO 2022 World Congress}}\end{large} \\
\faCalendar ~ \textsf{December, 2022} \hfill \faGlobe ~ \textsf{Cancun, Mexico} \\
HUPO is the Human Proteome Organization which organizes a world congress somewhere around the world each year.
During these conferences, lectures by world renowned speakers and exciting networking opportunities are offered.
I presented a poster about proteogenomics analysis using the Unipept Desktop application at this conference and participated in a discussion at the metaproteomics-specific session.

\pagebreak

\begin{large}\textbf{\textsf{EuBIC Developers' Meeting 2023}}\end{large} \\
\faCalendar ~ \textsf{January, 2023} \hfill \faGlobe ~ \textsf{Ascona, Switzerland} \\
The EuBIC-MS Developers Meeting is organized every other year.
This meeting is aimed at bringing together computer scientists and developers in the field of mass spectrometry-related bioinformatics to discuss and work together in an open and constructive spirit.
The program is split between keynote lectures and multiple hackathon sessions where the participants develop bioinformatics tools and resources addressing outstanding needs in the mass spectrometry-related bioinformatics and user community.

Together with Tibo Vande Moortele and Tim Van Den Bossche, I hosted a hackathon session at this conference entitled ``Exploring and solving functional analysis gaps in metaproteomics``.
During this week, we've had a lively discussion with the members of our hackathon team to discuss these functional analysis gaps in metaproteomics and settled on improving support for highlighting taxonomic diversity of metaproteomic samples in metabolic pathways.
We started the development of a web application that allows users to upload a list of peptides, which will be taxonomically analysed.
All identified taxa will be highlighted on a metabolic pathway by using the KEGG database.

\newpage
