\chapter*{Samenvatting}

Eiwitten zijn alomtegenwoordig.
Ze vormen een essentieel onderdeel voor alle leven op aarde en staan in voor een heleboel levensbelangrijke functies.
Zo zorgen enzymen er bijvoorbeeld voor dat ons lichaam op een efficiënte manier voedingsstoffen kan omzetten in energie en andere bouwstenen.
Of zorgen onze hormonen ervoor dat verschillende cellen en organen met elkaar kunnen communiceren en essentiële levensprocessen kunnen reguleren.
Niet alleen mensen, maar elk levend organisme op onze planeet steunt op de goede werking van eiwitten voor z'n voorbestaan.

Eiwitten bestaan zelf uit kleinere bouwstenen die we aminozuren noemen.
Deze moleculen zweven rond in ons lichaam en worden aan elkaar gekoppeld om een eiwit met een bepaalde functie en structuur te kunnen opbouwen.
De exacte volgorde van deze aminozuren bepaalt waarvoor het eiwit gebruikt kan worden en ligt vast in ons DNA.
Sinds de eerste ontdekking van eiwitten in de natuur, is er reeds een heleboel onderzoek naar deze grote moleculen uitgevoerd en hebben onderzoekers een grote databank opgesteld waarin ze voor elk gekend eiwit bijhouden bij welk organisme het hoort en wat zijn functie is (indien gekend).

Dankzij de massaspectrometer, een erg complex en duur toestel, is het mogelijk om te achterhalen welke eiwitsequenties er voorkomen in een bepaalde omgeving (zoals het bloed, de stoelgang, de grond rondom een bepaalde plant, etc.) wat ons daarna ook weer in staat stelt om deze eiwitsequenties op te zoeken in de bestaande eiwitdatabank.
Door elk van de geïdentificeerde eiwitsequenties op te zoeken in deze databank, kunnen we een rapport opstellen met de organismen die voorkomen in de onderzochte omgeving en welke functies ze daar mogelijks aan het uitvoeren zijn.

Unipept is een stuk software dat precies dit doet.
Onderzoekers dienen een lijstje van geïdentificeerde eiwitsequenties op te geven, waarna deze stuk-voor-stuk opgezocht worden door Unipept in een eiwitdatabank.
Naderhand worden al deze resultaten geaggregeerd en presenteert Unipept een mooi overzicht aan de gebruikers waarin alle geïdentificeerde organismen, en welke functies ze op dat moment uitvoeren, aanwezig zijn.

Over de laatste jaren zijn de stalen die onderzoekers willen analyseren met Unipept steeds groter geworden waardoor het niet langer mogelijk is om hiervoor een webapplicatie te gebruiken.
In hoofdstuk \ref{chapter:unipept_desktop_v1} kan u lezen hoe ik dit probleem heb aangepakt door de Unipept Desktop applicatie te ontwikkelen.
Dit is een software applicatie die niet langer door een browser wordt uitgevoerd, en het mogelijk maakt om veel grotere eiwitstalen te verwerken.
