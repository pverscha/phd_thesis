\chapter*{Acknowledgements}

\parbox{10cm}{
Now my life is better because you're a part of it \\
Yes, I know without you by my side that I would be different \\
Yes, I feel my life is better \\
And so is the world we're livin' in \\
I'm thankful for the time I spent with my best friends
}
\begin{flushright}\faMusic\ \ Best Friend - Jason Mraz\end{flushright}

\vspace{0.25cm}
\begin{center}
    $\sim$
\end{center}
\vspace{0.25cm}

Soms zegt een lied zoveel meer dan een muur aan tekst, maar ik wil het hier toch proberen, gewoon omdat er zoveel dingen zijn die ik even op papier wil zetten en wil delen met de hele wereld.
Het klinkt misschien cliché, en misschien is het dat ook wel, maar ik had nooit (maar echt nooit) op dit punt kunnen geraken zonder alle mensen die rond me staan.
Ik denk dat ieder van jullie wel weet dat de afgelopen vier jaar op sommige momenten erg moeilijk waren.
En het is op deze momenten dat ik ook heb leren beseffen wat het betekent om zo een grote groep van allemaal verschillende, maar even fantastische mensen te leren kennen hebben.

In de eerste plaats wil ik mijn fantastische gezin bedanken.
Mama, papa en Maarten, we hebben samen een heleboel ongelofelijk mooie momenten meegemaakt.
Onze vele reizen samen in de zomer, onze wekelijkse uitstapjes doorheen het hele land, de (soms erg grote) uitdagingen die ons allemaal nog dichter bij elkaar hebben gebracht en die zondagmiddagen waarop er altijd wel weer iemand van ons vier toch even op uit wou trekken (hier denk ik vooral aan jou, mama).

Mama, dankzij jou ben ik de ``wetenschappelijke tour" opgegaan.
Je ongelofelijk aanstekelijke interesse in alles wat te maken heeft met wetenschap, biologie, fysica en wiskunde heeft me gefascineerd, al sinds ik een paar jaar oud was.
Onderliggend is dit ook de reden dat ik vandaag doe wat ik doe.
En dat is ook wat ik graag doe.
Toen je ziek werd, je moest stoppen met werken en je er toch nog voor gekozen hebt om opnieuw een vak te volgen aan de universiteit toont nog maar eens dat je nooit kon stilzitten en dat je ervoor zorgde dat elk moment in je leven de moeite waard was.
Je bent zolang vooruit blijven kijken en je bent ons steeds zoveel wondermooie perspectieven blijven bieden, ookal werd dat er voor jou niet makkelijker op.
De manier waarop je zo trots naar me stond te kijken toen je kwam kijken naar mijn proclamatie hier aan de UGent ga ik nooit meer vergeten.
Dankjewel!
Ik mis je elke dag een beetje meer, en ook al zeggen ze dat tijd alles slijt, ga je voor mij altijd blijven betekenen wat je hebt betekend.

Papa, jij telt voor twee.
De laatste 2 jaar waren erg speciaal, voor ons allemaal, maar jij hebt zowel Maarten als mij er zo mooi doorgetrokken.
De reizen die we de laatste jaren met ons drie hebben gemaakt zijn nog altijd een van de hoogtepunten van het jaar en dat zal ook zo blijven.
Ik kijk er al naar uit om in oktober samen de Verenigde Staten te verkennen en weer een heleboel nieuwe avonturen tegemoet te gaan.
Zoals jij er voor mij en Maarten was de laatste 26 jaar, gaan wij er ook altijd voor jou zijn.
Wanneer mama even een moeilijk moment had, heb je ons opgevangen en geprobeerd er toch nog steeds het beste van te maken (en altijd met succes!).

Ook Maarten vergeet ik zeker niet.
Al vanaf je een paar jaar oud was, zijn we samen een kamp beginnen bouwen in de tuin en begonnen we brood te roosteren op een fakkel in de grond.
Ook al hebben we ons kamp weer afgebroken en roosteren we geen brood meer, we gaan nog altijd samen weg en verkennen de wereld stap-voor-stap.
Net zoals alle broers, hebben we ook af en toe eens een discussie, maar duren die dan ook nooit lang.
Laat ze daar in Brussel maar eens zien hoe ze hun ``mobiliteit" kunnen regelen! :)

Ook al kunnen ze niet praten, ik wil toch ook onze trouwe viervoeters bedanken die steeds op m'n toetsenbord liepen, op m'n arm lagen, snurkgeluiden zaten te produceren of om eten kwamen smeken tijdens de voorbije vier jaar.
Mimi, Tita, Poockey, Blackie en Mila (nee, er bestaat niet zoiets als ``te veel" katten), bedankt voor alle knuffels!

Ik denk dat iedereen in Belsele het wel zal beamen, de familie Ongena is uniek en ik ben zo blij dat ik er een deel van uitmaak!
Nonkel Jef, tante Trees en tante Lieve, bedankt!
Ook bij jullie ontbreekt het zich niet aan interesse in de wetenschap en enorm veel hulp bij mijn vier jaar lange tocht tijdens mijn doctoraat aan de UGent.

Aan de hoeveelheid kleren die ik vuilmaak bij het eten en het feit dat ik ook zo graag van de kleine dingen in het leven geniet, wordt al snel duidelijk dat er net zoveel Verschaffelt-bloed door m'n aderen stroomt.
Nonkel Frank, bedankt om altijd naar mijn verhalen te luisteren wanneer ik weer eens van een reisje (al dan niet voor het werk) naar huis kom.
Bedankt ook om altijd zo positief in het leven te staan en nergens een probleem van te maken, zoiets is niet iedereen gegeven.

Naast mijn familie wil ik ook graag mijn fantastische vrienden bedanken.

Anaïs en Hanne, jullie zijn ondertussen al meer dan vrienden geworden.
We kennen elkaars familie door-en-door en ik denk dat we veilig kunnen stellen dat we ook van elkaar meer en meer een familie worden.
Ik kan altijd bij jullie terecht, met alles, en het gevoel dat er zo een fantastisch paar vrienden altijd voor je klaarstaat is onbeschrijfelijk.
Bij jullie mag ik altijd klagen, zagen, en om raad vragen, maar kunnen we evengoed ook genieten van prachtige momenten.
Elk van onze reisjes is nog leuker dan het vorige (is dat zelfs nog mogelijk?) en ik denk dat we samen ondertussen al de halve wereld hebben verkent.
Wanneer ik weer eens de kluns uithang en we ons bijna doodlachen of wanneer we geïmproviseerde karaoke houden in de auto, besef ik hoe mooi het leven wel is.
Dankjewel!
Op naar nog zoveel meer ``good times" (and getting scammed on a boat)!

Niko, op jou kan ik altijd rekenen wanneer ik weer eens niet weet hoe ik met een nieuwe technologie moet werken, wanneer ik eens goed wil gaan eten in een restaurant (een goede pasta, of een avond mezze bij de Griek gaan er altijd in).
Vanaf het moment dat we aan de opleiding informatica gestart zijn, hebben we al onze groepswerken samen gemaakt en tot op vandaag werken we beide ook nog steeds aan de UGent.
Als ik iemand zoek om samen een toezicht te doen tijdens een examen, en dan misschien op hetzelfde moment ook een reis te boeken, dan weet ik aan welk adres ik moet zijn.
De herinneringen aan de roadtrip door Bulgarije, Griekenland en Noord-Macedonië samen met Jarre die daarvan het gevolg was, ga ik voor altijd koesteren.

Wouter, al vanaf het moment dat we samen in dezelfde kleuterklas in hartje Belsele terechtkwamen zijn we beste vrienden.
Nu, bijna 24 jaar later, wonen we nog steeds op 2 km van elkaar en komen we elkaars tuin regelmatig ``inspecteren".
Ik ben er zeker van dat we elkaar binnen 50 jaar nog komen opzoeken en we ook dan nog altijd even goed overeen gaan komen.

Niko, Jarre, Rien, Louise, Jorg, Sam, Nils, Arne en Sander; iedereen die ik aan de UGent heb leren kennen tijdens die 5 jaar informatica studeren en alle leuke activeiten en discussies die we samen hebben gehad: welgemeende bedankt!
Op nog vele kampeerweekends en gezellige avonden samen!

Aan al m'n bureaugenoten, collega's en vrienden van de S9, bedankt!
Rien, Niko, Bart, Tibo, Charlotte, Toon, Tom, Jorg, Wout, Jonathan, Louise, Felix, Heidi, Steven, Alexis, Robbert, Asmus: bedankt voor de leuke (en soms net iets té lange) koffiepauzes, lunchpauzes en spelletjesavonden!

Aan al m'n collega's en vrienden van CompOmics in Zwijnaarde, Alireza, Arthur, Caro, Daria, Enrico, Jasper, Kevin, Natalia, Nina, Pathmanaban, Patricia, Ralf, Robbin, Sven, Tanja, Tim, Tine en Toon, dikke merci!
Met z'n allen op conferentie naar Lissabon of Mexico is altijd een ongelofelijke ervaring.
Maar ook samen door Gent trekken en iets gaan eten, drinken of een avondje karaoke doen is bijna verslavend.
Tim en Ralf, ik denk nog elke dag aan onze reis met Tanja en Yannek door Mexico.
Wat een zalige ervaring was dat!
Ik begin nog spontaan te lachen wanneer ik denk aan onze ``boysband" en hoe we door Yucatan werden begeleid door onze tourmanager Tanja!

Laatst in dit Nederlandstalige deel van mijn dankwoord, maar zeker niet minst, wil ik zeker mijn promotoren Peter, Bart en Lennart bedanken!
Ik wil jullie ongelofelijk hard bedanken voor jullie geloof in mij, jullie waardevolle steun wanneer ik die nodig had en jullie enorme begrip wanneer de dingen niet liepen zoals ze zouden moeten lopen.
Ik kijk er naar uit om de komende jaren verder met jullie te kunnen samenwerken en weer nieuwe dingen te ontwikkelen en ontdekken.
Bart, ik wil jou gerust vertegenwoordigen op de verschillende conferenties overal, met veel plezier zelfs :P (ik zou liegen als ik zei dat ik wil proberen om wat minder conferenties mee te pikken).
En Lennart, nog een kleine opmerking, ik ga nooit meer vergeten dat er 20 verschillende aminozuren bestaan (en geen 22) ;)

Now, a switch to English in order to allow me to thank all of the friends that I met during the last four years on conferences, research stays or travels.

A big thank you to all of my German friends Tanja, Yannek, Sasan and Caro!

Tanja, thank you sooo much for inviting me to Berlin to work together on the amazing Peptonizer2000!
I hope it will finally be fully functional some day, and that Unipept will be finally be able to handle the Peptonizer's massive requests ;)
Maybe we can even distribute some more stickers then!
My stay in Berlin was wonderful and I still miss the moments we've had there together with Yannek, Sasan and the others at the BAM e-science group.
I will never forget our holidays together.
And remember: sleep tight, don't let the bed bugs bite.
I'm already curious as to what you have planned for the coming years.
Keep the Tanja-dollars coming and please come back to Ghent for a few months!
We've had so much fun during those three months that you spent here last year.

Yannek, together with Tanja we've had the best time before, after or during the various conferences that we attended all-over the world.
Keep inventing new cocktail recipes and thinking of innovative pranks that you can pull on me (or Tanja).
I hope they're equally good as the passata-based ones you invented before going to Mexico.
I will gladly be a guinea pig and try all of them out.

Caro, I've had a lot of fun during the three months you spent at CompOmics in Ghent.
Please don't forget your Nintendo Switch or PlayStation for the next conference, I'm already looking forward to the evening Mario Kart and Singstar sessions!

I was trying to keep all of these acknowledgements as brief as possible, but I have clearly failed.
Please forgive me if I forgot someone, you're all equally important to me!

Thank you all sooo soo much!
And I hope to be able to have fun with all of you for many years to come.
